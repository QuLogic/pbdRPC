
\section[An Application with \pkg{pbdCS}]{An Application with \pkg{pbdCS}}
\label{sec:application_cs}
\addcontentsline{toc}{section}{\thesection. An Application with \pkg{pbdCS}}

Similar to the \pkg{remoter}, the \pkg{pbdCS}~\citep{pbdCS} provides
interactivity for clusters running \proglang{R}'s via the
\pkg{pbdMPI}~\citep{Chen2012pbdMPIpackage} in SPMD computing
framework~\citep{pbdR2012,hpsc2012}.
See \citet{xsede16} for an introduction of \pkg{remoter} and \pkg{pbdCS}, and
see \url{https://github.com/snoweye/user2016.demo}
for a demo of both packages.

In a simplified scenario such as the setting in
Section~\ref{sec:handling_login_information},
several pbdCS \proglang{R}'s can run 4 instances on the server,
Xubuntu, ip=192.168.56.101 as the example below.
\begin{Command}[title=\pkg{pbdCS} cluster with 4 \proglang{R} instances]
$ source ~/work-my/00_set_devel_R
$ nohup mpiexec -np 4 Rscript -e 'pbdCS::pbdserver()' > .cclog 2>&1 < /dev/null &
$ ps ax|grep '[p]bdCS::pbdserver'
$ kill -9 $(ps ax|grep '[p]bdCS::pbdserver'|awk '{print $1}')
\end{Command}
The example above is very similar to the one in
Section~\ref{sec:application_rr}, but further demonstrates
how to ``start'', ``check'', and ``kill''
a \pkg{pbdCS} cluster with 4 \proglang{R} launched by/within
the MPI program \code{mpiexec}.

In an well established server, one can use \code{ssh} or \code{plink.exe}
to send those commands from the local laptop.
Furthermore, one may also use \pkg{pbdRPC} directly within an \proglang{R}
environment to send those commands. The code below shows the example.
\begin{Code}[title=Using \pkg{pbdRPC} to control \pkg{pbdCS}]
> library(pbdRPC, quietly = TRUE)
>
> ### Alter login information as needed
> # rpcopt_set(user = "snoweye", hostname = "192.168.56.101")
> m <- machine(user = "snoweye", hostname = "192.168.56.101")
> .pbd_env$RPC.CT$use.shell.exec <- FALSE
>
> preload <- "source ~/work-my/00_set_devel_R; "
> start_cs(m, preload = preload)
character(0)
>
> library(pbdCS)
> pbdCS::pbdclient(addr = "192.168.56.101")

pbdR> library(pbdMPI)
pbdR> allreduce(1)
[1] 4 
pbdR> q()
>
> check_cs(m)
[1] "12578 ?        Sl     0:00 mpiexec -np 4 Rscript -e pbdCS::pbdserver()"                                                      
[2] "12580 ?        Sl     0:00 /home/snoweye/work-my/local/R-devel/lib64/R/bin/exec/R --slave --no-restore -e pbdCS::pbdserver()"
[3] "12581 ?        Sl     0:00 /home/snoweye/work-my/local/R-devel/lib64/R/bin/exec/R --slave --no-restore -e pbdCS::pbdserver()"
[4] "12583 ?        Sl     0:00 /home/snoweye/work-my/local/R-devel/lib64/R/bin/exec/R --slave --no-restore -e pbdCS::pbdserver()"
[5] "12588 ?        Sl     0:00 /home/snoweye/work-my/local/R-devel/lib64/R/bin/exec/R --slave --no-restore -e pbdCS::pbdserver()"
> kill_cs(m)
character(0)
\end{Code}
where \code{pbdclient()} is for connect to the \pkg{pbdCS} cluster
started by \code{start_cs()}.

The \code{start_cs()}, \code{check_cs()}, and \code{kill_cs()} are
all wrapper functions of \code{srpc()} to submit different commands stored in
\code{.pbd_env$RPC.CS$start}, \code{.pbd_env$RPC.CS$check}, and
\code{.pbd_env$RPC.CS$kill}, respectively.
The details of \code{RPC.CS} are in the example below.
\begin{Code}[title=\code{RPC.CS} for controlling \pkg{pbdCS}]
> .pbd_env$RPC.CS
$check
[1] "ps ax|grep '[p]bdCS::pbdserver'"

$kill
[1] "kill -9 $(ps ax|grep '[p]bdCS::pbdserver'|awk '{print $1}')"

$start
[1] "nohup mpiexec -np 4 Rscript -e 'pbdCS::pbdserver()' > .cslog 2>&1 < /dev/null &"

$preload
[1] "source ~/work-my/00_set_devel_R; "

\end{Code}

