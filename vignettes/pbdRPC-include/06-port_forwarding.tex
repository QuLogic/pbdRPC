\section[Local Port Forwarding]{Local Port Forwarding}
\label{sec:local_port_forwarding}
\addcontentsline{toc}{section}{\thesection. Local Port Forwarding}

The \pkg{remoter} command \code{client()} has a default setting
to connect to the \pkg{remoter} server using \code{addr = "localhost"} and
\code{port = 55555} which assumes the \pkg{remoter} server and client are
both working at \code{localhost}.
This may only be possible for convenience of development and debugging only.
In general, the server can be anywhere and more powerful than a laptop.
Again, We may consider the environment setup in
Sections~\ref{sec:application_rr} and \ref{sec:application_cs} to
demonstrate local port forwarding, even though the setup is over simplified
it is quite common for most general users.
The server is running at \code{192.168.56.101:55555}, so the argument
\code{addr = "192.168.56.101"} in the \pkg{remoter}
command \code{client(addr = "192.168.56.101")} from the \code{localhost}
is necessary.

Note that this above case may not be a good reason to show local port
forwarding.
However, it can avoid typing address or to be independent to the \code{addr}.
One may consider to forward the \code{localhost} port \code{55555} to the
server directly.

The following code serves the purpose of local port forwarding in
\pkg{pbdRPC} using \code{rpc()}, then start a \pkg{remoter} server and
launch a connection via \code{client()} without changing arguments.
\begin{Code}[title=Forward \code{localhost:55555} to \code{192.168.56.101:55555}]
> library(pbdRPC)
> rpc(args = "-N -T -L 55555:192.168.56.101:55555")
>
> library(remoter)
> start_rr()
> client()    # equivalent to client(addr = "192.168.56.101")
\end{Code}

First, \code{rpc(args = "-N -T -L 55555:192.168.56.101:55555")}
forwards the connection between \code{55555} of the local host and
\code{192.168.56.101:5555}.
Note that this call (local process) is running in background and is
no disconnected even after quiting \proglang{R} ,
because \code{intern = FALSE} and \code{wait = FALSE} are set by default
in \code{rpc()}.
{\color{red}
The additional command ``\code{kill -p [pid]}'' may be needed to manually
kill the local process (pid) when the forwarding is not needed anymore.
}
See Section~\ref{sec:arguments_local_port_forwarding} or
\code{ssh}'s man page for details of arguments \code{-N -T -L}
(inside \code{args}) to the \code{ssh} or \code{plink.exe}.

Second, \code{client()} tries to connect with \code{localhost:55555}
by default because it is from the laptop. The connection is then
redirected to \code{192.168.56.101:5555} as well because the local
port is being forwarded.


\subsection[Arguments for Local Port Forwarding]{Arguments for Local Port Forwarding}
\label{sec:arguments_local_port_forwarding}
\addcontentsline{toc}{subsection}{\thesubsection. Arguments for Local Port Forwarding}

Note that the \code{ssh} and \code{plink.exe} has similar functionalities
for local port forwarding.

The argument \code{-L} in \code{ssh} or \code{plink.exe} is a typical option
for local port forwarding. The usage from the man page of the \code{ssh} says
\begin{Command}[title=From \code{ssh} man page]
-L [bind_address:]port:host:hostport
      Specifies that the given port on the local (client) host is
      to be forwarded to the given host and port on the remote side.
\end{Command}

Because the call of local port forwarding
needs to be either alive or active during the access
of other applications to the \code{[bind_address:]port},
two other useful arguments are \code{-N} and \code{-T} that can combine and
use with local port forwarding. The usages of both arguments
from the man page say
\begin{Command}[title=From \code{ssh} man page]
-N    Do not execute a remote command.  This is useful for just
      for warding ports (protocol version 2 only).
-T    Disable pseudo-terminal allocation.
\end{Command}
i.e. batch and background modes are preferable.

